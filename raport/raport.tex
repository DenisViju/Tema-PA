\documentclass{article}
\usepackage{amsmath}
\usepackage{graphicx}
\usepackage{algorithmicx}
\usepackage{algpseudocode}
\usepackage{hyperref}



\title{Lobster Fisherman Optimization}
\author{VIJULIE DENIS-EMANUEL}
\date{Mai 2024}

\begin{document}

\maketitle

\section{Link către Repository}
Repository-ul cu codul sursă poate fi găsit la adresa:
\url{https://github.com/DenisViju/Tema-PA.git}

\section{Introducere}
Scopul acestui proiect este de a dezvolta un software care să permită selectarea optimă a homarilor pentru a maximiza valoarea totală a capturii, respectând limita de capacitate a plasei.

\section{Enunțul Problemei}
Un pescar explorează o regiune de coastă bogată în homari, fiecare având propria sa dimensiune și valoare. Plasa pescarului are o capacitate limitată, exprimată în numărul total de centimetri pe care îi poate conține. Având o listă detaliată cu dimensiunile și valorile homarilor disponibili în acea regiune, sarcina este de a elabora o strategie prin care pescarul să selecteze homarii astfel încât să maximizeze valoarea totală a capturii, respectând în același timp limita de capacitate a plasei.

\section{Algoritmi}
Algoritmul folosit pentru această problemă este o variantă a problemei rucsacului (Knapsack problem) și este implementat folosind programarea dinamică.

\begin{algorithmic}[1]
  \Procedure{DynamicProgramming}{Lista homarilor $H$, capacitatea plasei $C$}
  \State $n \gets$ numărul de homari din lista $H$
  \State Initializează o matrice $V[0..n][0..C]$ pentru a stoca valorile maxime
  \For{$i\gets 0$ to $n$}
    \For{$j\gets 0$ to $C$}
      \If{$i=0$ sau $j=0$}
        \State $V[i][j] \gets 0$
      \ElsIf{$w_i \leq j$}
        \State $V[i][j] \gets \max\{V[i-1][j], v_i + V[i-1][j-w_i]\}$
      \Else
        \State $V[i][j] \gets V[i-1][j]$
      \EndIf
    \EndFor
  \EndFor
  \State $S \gets$ lista homarilor selectați
  \For{$i\gets n$ down to $1$}
    \If{$V[i][C] \neq V[i-1][C]$}
      \State Adaugă homarul $i$ la lista $S$
      \State $C \gets C - w_i$
    \EndIf
  \EndFor
  \State \Return $S$
  \EndProcedure
\end{algorithmic}


\section{Date Experimentale}
Datele experimentale au fost generate aleator folosind un program C care produce seturi de date non-triviale mari și foarte mari.

\section{Proiectarea Experimentală a Aplicației}
Aplicația a fost proiectată pentru a fi modulară și ușor de utilizat, cu un cod sursă bine structurat și comentat.

\section{Rezultate și Concluzii}
\section{Date Experimentale}
Datele experimentale au fost generate utilizând un program C care produce seturi de date non-triviale. Aceste date au fost generate pentru a simula dimensiunile și valorile homarilor disponibili într-o regiune de coastă. Mai jos sunt prezentate câteva exemple de date generate:

\begin{itemize}
  \item Lobster 1: Dimensiune = 2 cm, Valoare = 10 monede de aur
  \item Lobster 2: Dimensiune = 8 cm, Valoare = 29 monede de aur
  \item Lobster 3: Dimensiune = 4 cm, Valoare = 37 monede de aur
  \item Lobster 4: Dimensiune = 4 cm, Valoare = 20 monede de aur
  \item Lobster 5: Dimensiune = 8 cm, Valoare = 2 monede de aur
  \item Lobster 6: Dimensiune = 4 cm, Valoare = 35 monede de aur
  \item Lobster 7: Dimensiune = 1 cm, Valoare = 4 monede de aur
  \item Lobster 8: Dimensiune = 3 cm, Valoare = 30 monede de aur
  \item Lobster 9: Dimensiune = 2 cm, Valoare = 17 monede de aur
  \item Lobster 10: Dimensiune = 7 cm, Valoare = 48 monede de aur
\end{itemize}

Aceste date au fost utilizate pentru testarea algoritmului de selecție a homarilor.

Rezultatele experimentale arată că algoritmul de programare dinamică este eficient pentru optimizarea valorii capturii, maximizând valoarea totală fără a depăși limita de capacitate.

\end{document}
